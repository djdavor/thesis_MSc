\section{Literature Review}

\subsection{Employee Stock Options} %RED
%definition, graph to explain timeframe

    characteristics: non-transferability of options + restrictions on short-selling underlying stock (for hedging purposes)
    We need some terminology here: an option is in-money when \dots, American vs European \dots, 

%numerical facts and graph that shows evolution over time
    "In 1992, firms in the S\$Ps 500 granted their employees options worth a total of \$11 billion at the time of grant; by 2000, option grants in S\&P 500 firms increased to \$119 billion" \cite{hall2003trouble}

%incentives
    Risk-averse executives are offered options so to undertake more risk than they would normally → take more risk and thus higher ROI + act as counterbalance for risk aversion of executives
    Linking realized compensation with stock performance (for listed companies)
    Issue with systematic vs idiosyncratic risk  \cite{armstrong2012executive} \cite{heron2017stock}
        employees feel they can predict/influence idiosyncratic rather than systematic uncertainty
        this affects also difference in valuation between executive value and cost of option \cite{meulbroek2001efficiency}

    Factors driving adoption of ESO plan: ownership concentration, liquidity, CEO and institutional ownership, investment intensity, historical market return \cite{pasternack2002factors}
    \cite{ross2004compensation} takes a strong standpoint arguing that NO incentive scheme will make all agents more or less risk averse uniformly
    It remains however incomplete talking about incentives without considering the whole comp package

%early exercise: show some graph
    Volatility of stock inversely related to early exercise: the medium is executive's subjective valuation, that increases in underlying's volatility \cite{heron2017stock} \cite{izhakian2017risk} + Others
    Importance of ambiguity - see down below 
    A bit outdated but still... \cite{huddart1996employee}
    Why: diversification vs liquidity concerns \cite{murphy2019employees}


%possible critics
    Bunch of other benefits: attract motivated and entrepreneurial employees, do not expend cash immediately, retention of employees  \cite{hall2003trouble}
    BUT, NO evidence on convexity of options in \cite{hayes2012stock}

\subsection{Dynamic ESOs/reload options} %RED
    This is where the idea of reload/dynamic options kicks in: recoup some "time value" sacrificed when exercising early - see also \cite{hall2002stock} for formal analysis of subjective value vs company cost
    They allow to recover the time value of options (ie, the value foregone when exercising the option): it is defined as the option price minus the intrinsic value (underlying - strike price if in-money, otherwise) at time $t$; the intuition is that the underlying could grow even more and hence, in that case, by exercising later one could make a larger profit. This possibility is forgone by the option holder that exercises before ,maturity.

%reload options
    Also referred to as restoration or replacement options; they are call option that grant new options once exercised
    Reload option exercised when in-money; it is exotic option; fee structure \cite{zhang2010knightian}
    See \cite{hemmer1998optimal} and \cite{hemmer2000reload}
    First used by Cook for Norwest Corporation
    data + valuation model - see \cite{dybvig2003employee}
    + encourage stock ownership 
    heterogeneity in how they have been implemented - see Frederic W. Cook and Company (1998) and Hemmer, Matsunaga, and Shevlin (1998) 
        infinite reloads: not an issue because in any case the value is between the American option (LB) and between having granted a stock directly (as the UB) \cite{dybvig2003employee}
            LB: just follow the American exercise policy and never exercise the reloaded option
            UB: recursive example; strict inequality when stock grants dividends, bc they won't be enjoyed by the reload option holder 
            they use infinite reloads but I will focus on only one reload 
    -> illustrate with an example


%dynamic ESOs 
    \cite{huang2013dynamic}:
        definition and 
        advantages: very many
        drawbacks: only accounting cost
        valuation used for DESOs

%issue with valuation
    see above valuations 
    \cite{carpenter1998exercise} finds that a simpler extension of the base model may be more useful than the intricated preference-based (utility-maximizing) model
    classical valuation vs utility-based approach show important differences (e.g., \cite{lau2005valuation} for reload options; \cite{ingersoll2006subjective} for more general case)


 



\subsection{Empirics on (executives') risk aversion} %ORANGE

    Has clearly an effect on early exercise \cite{izhakian2017risk} \cite{murphy2019employees}
    See also effect on block exercise in \cite{grasselli2009risk}





\subsection{Continuous time principal-agent models} %YELLOW
    See \cite{cvitanic2013dynamics} for a good introduction
%Discrete time - see Sannikov 2008
%Shift to continuous-time
%Stream started by Sannikov
%See Master Thesis for a good overview of these dynamic models




