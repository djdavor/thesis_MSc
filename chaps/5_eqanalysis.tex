\section{Equilibrium analysis}







Goal: show how allocations change within the three different equilibria: full info, moral hazard, moral hazard + adv. selection 



%Optimal exercise: in the form of exercise policy, which is done by studying the so-called continuation region of the employee (ie, where the employee would continue to hold the option)


\subsection*{Description}
\cite{dai2005valuing} propose using N=10,000 as a good asymptotic approximation.

Stock price Simulation
    points and paths; horizon of 20 years (sufficient for R + reload option)
    default parameters + additional effort and volatility
    delta goes from 0 to 1

Firm parameters
    alphas and gammas it can offer
    beta goes from 0 to 1 (large and small firm)

Agent
    default is 67/33 agent; then also 50/50 agent
    a_h and a_l + sigma_h and sigma_l; low set to 0
    key parameters are y_RN and y_R1,y_R2; see exercise years from empirical literature


First, we simulate the paths under the different parameters, which is mainly influenced by the level of effort and volatility + stock price evolution accounts for whether the option is still to be exercised or has been exercised.




+ we label the type of equilibrium, for each 1st, 2nd, or 3rd best:
## Shutdown: 1 agent, 2 agents
## Pooling: Same a and same sigma
## Separating: else (either a or sigma different, both agents active)

\subsection*{Simulation Results} 
Base case - set fixed (justified) parameters
HOW TO DISPLAY THEM??

%Different:
    FIRM: n_agents + lamb, beta, larger alphas and gammas set

    AGENT: U_bar, delta, (rho_L/H, a_L/H, sigma_L/H) - absolute value and distance, y_RN/R1/R2

    +: agent's different outside wealth

    -> WHAT equilibrium gets selected? Is it first, second, third best? What's the value difference from the other type of equilibria? How sensitive it is to the change of parameters?


%Consider wealthy CEO vs manager at some Internet-based firm (\cite{carpenter1998exercise}) - the former has lower portion of wealth invested in options hence has the "risk-adjusted" risk-aversion lower than the latter. Alternatively, differences in valuation are reflective of differences in wealth composition rather than risk aversion alone!! \cite{henderson}  


\subsection*{Comparison with RN case only}
ie, we set alphas = [1.0] and gamma = [0.0] and compare the results with the previous case.