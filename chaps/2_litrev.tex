\section{Literature Review}

\subsection{Employee Stock Options} %RED
%definition, graph to explain timeframe

%numerical facts and graph that shows evolution over time
    "In 1992, firms in the S\$Ps 500 granted their employees options worth a total of \$11 billion at the time of grant; by 2000, option grants in S\&P 500 firms increased to \$119 billion" \cite{hall2003trouble}

%incentives
    Risk-averse executives are offered options so to undertake more risk than they would normally → take more risk and thus higher ROI + act as counterbalance for risk aversion of executives
    Linking realized compensation with stock performance (for listed companies)
    Issue with systematic vs idiosyncratic risk  \cite{armstrong2012executive} \cite{heron2017stock}
        employees feel they can predict/influence idiosyncratic rather than systematic uncertainty
        this affects also difference in valuation between executive value and cost of option \cite{meulbroek2001efficiency}

    Factors driving adoption of ESO plan: ownership concentration, liquidity, CEO and institutional ownership, investment intensity, historical market return \cite{pasternack2002factors}

%early exercise: show some graph
    Volatility of stock inversely related to early exercise: the medium is executive's subjective valuation, that increases in underlying's volatility \cite{heron2017stock} \cite{izhakian2017risk} + Others
    Importance of ambiguity - see down below 
    A bit outdated but still... \cite{huddart1996employee}
    Why: diversification vs liquidity concerns \cite{murphy2019employees}


%possible critics
    Bunch of other benefits: attract motivated and entrepreneurial employees, do not expend cash immediately, retention of employees  \cite{hall2003trouble}
    BUT, NO evidence on convexity of options in \cite{hayes2012stock}

\subsection{Dynamic ESOs/reload options} %RED
    This is where the idea of reload/dynamic options kicks in: recoup some "time value" sacrificed when exercising early - see also \cite{hall2002stock} for formal analysis of subjective value vs company cost

%reload options
    See \cite{hemmer1998optimal} and \cite{hemmer2000reload}
%dynamic ESOs 
    \cite{huang2013dynamic}:
        definition and 
        advantages: very many
        drawbacks: only accounting cost
        valuation used for DESOs

%issue with valuation
    see above valuations 


 



\subsection{Empirics on (executives') risk aversion} %ORANGE

    Has clearly an effect on early exercise \cite{izhakian2017risk} \cite{murphy2019employees}





\subsection{Continuous time principal-agent models} %YELLOW
    See \cite{cvitanic2013dynamics} for a good introduction
%Discrete time - see Sannikov 2008
%Shift to continuous-time
%Stream started by Sannikov





%%TO-DO NEXT:
%%%Check valuation folder
%%%Check unclassified folder quickly
%%%Check theory folder