\section{Introduction}
%Catchphrase


%Intro to topic + why it matters

ESOs have become a popular form of compensation for executives in the past decades as a tool to align the interests of the employee with those of the shareholders. 



%Chapter outline (2/3 phrases per chapter)

In the first chapter, we present an overview of the relevant streams of literatures: ESOs and reload options, valuations, risk aversion and principal-agent problem. We first discuss the (recent) history of ESOs and their role in the executive compensation literature: as we will see, they have been increasingly used in a heterogeneous manner, with different features and terms, which has led to a variety of studies on the topic. We then focus on reload options --- options that at exercise give a new option, together with the usual exercise compensation --- and focus in particular on the idea of Dynamic Employee Stock Options, which are a type of reload options that have been proposed as a hybrid that allows for recovering the time value of traditional ESOs, together with benefits both for the executive and the firm. We then discuss the valuation of ESOs, which is a crucial aspect of the literature, as it allows to understand the incentives that ESOs provide, in a differential way to the firm and to the executive because of the inadequacy of the usual risk neutral framework in the latter case. Then, we discuss the more theoretical literature on risk aversion and the principal-agent problem, with its more recent applications to dynamic settings, which is the focus of our model.

In the second chapter, we build up the theoretical model that we will use throughout the paper. We use a continuous-time principal-agent problem with both adverse selection --- the coefficient of risk aversion of the executive --- and moral hazard --- the choice of effort and volatility of the executive. We have a risk-neutral firm that hires a risk-averse executive, who can choose between two types of options: a regular ESO, which gives only cash compensation when exercised, and a different option, which gives a part in cash compensation and a part in  new options at exercise. The executive can be either high or low risk-averse, and can exert high or low effort, and choose a project with high or low volatility. The firm can choose the parameters $(\alpha, \gamma)$ that determine the proportion of cash and options in the compensation. We discuss the 

In the third and fourth chapters we basically offer two points of view on the issue: in the third chapter, we focus on the valuation of the options, and in the fourth chapter we focus on the equilibrium analysis by means of some simple numerical simulations. In the third chapter, we develop a novel valuation framework for the risky option, that is able to account for both the recovered time value and additional compensation of this option. An important point is the difference in valuation between the firm and the executive, due to the risk aversion of the latter: indeed, while for the firm we can employ the traditional risk-neutral framework, for the executive we need to use a utility-maximization approach, that accounts for the risk aversion of the executive. We confirm the findings of \cite{carpenter1998exercise} that this indeed yields a difference in the valuations between the firm and the executive, which is relevant. Finally, we go a step further and study the incentives that the options provide to the executive, from both points of view. We find that \dots
In the fourth chapter, we developed a simple Python code that allows us to simulate the equilibrium of the model in a discrete-time framework. We find some contrasting results from before: without adverse selection or moral hazard, all agents choose the regular option and exert high effort, but once we introduce the more realistic scenario with unobserved type and actions the two agents continue to choose the regular option, but now with low effort and high volatility. These results are in general robust to changes in the parameters, except in two cases: for the average number of years until exercise, that we need to assume and impacts the final results, and to the difference in effort values, for which our code is not able to account for very small results. However, the difference in principal's utility between the two options is never very high, which leaves open the possibilities for future refinements of our model.




We develop valuation framework to study 
We propose a simple model to study this issue ...


\textit{To be written.}