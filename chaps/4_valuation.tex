\section{Valuation}
Valuation of options has been a great area of research, starting with Black-Scholes derivatives pricing and building upon it.
    Relevant for accounting statements and tax returns - FASB has changed valuation practices several times to ensure harmonization
    shape of valuation function → executive's exposure to risk and incentives
    valuation and hedge ratio/delta provide approximation of the incentive of option (which remain however limited and dependent on the full comp package!)


\subsection{Firm Cost} 
%Objective value

BS: what is it?
    allows for closed-form solution
    \begin{assumption}
        The rate of return on the riskless asset is constant and denoted as risk-free interest rate $r$.
    \end{assumption}
        
    \begin{assumption}
        The instantaneous log return of the stock price is an infinitesimal random walk with drift: the stock price follows a geometric Brownian motion, and it is assumed that the drift and volatility of the motion are constant.
    \end{assumption}

    \begin{assumption}
        The stock does not pay any dividends.
    \end{assumption}

    \begin{assumption}
        The market is arbitrage-free.
    \end{assumption}

    \begin{assumption}
        It is possible to borrow and lend any amount (even fractional) of cash at the riskless rate.
    \end{assumption}

    \begin{assumption}
        It is possible to buy and sell any amount (even fractional) of the stock.
    \end{assumption}
    
    \begin{assumption}
        There are no transaction fees.
    \end{assumption}
    
    Under these assumptions, \cite{black1973pricing} showed that it is possible to create a hedged position (i.e., a long position in the underlying stock and a long position in the option) whose value is independent of the price of the underlying. This dynamic hedging strategy leads to a PDE governing the price of the option, whose solution is given by the Black-Scholes formula.

    Note that the penultimate assumption assumes the possibility of short-selling, which in the case of ESOs may not be feasible. However, this is not always true, hence we use Black-Scholes as a good approximation. Note that not all of these assumptions are strictly necessary: extensions of this basic model are able to account for dynamic interest rates, transaction costs and taxes, and dividend payout.  


    BUT, BS can only be used with European option (options exercisable only at maturity). However, in our case we are dealing with American-stlye options, hence we resort to a different valuation methodology: binomial model.

BM: (see Battauz lecture notes + \cite{cox1979option} for a description)
    NOT closed-form solution (another approach would be Monte-Carlo or finite difference methods)
    lattice-based 


We wrote a short Python code to price our ESOs using the binomial model, and the Black-Scholes only for computing the intrinsic value of the option at exercise time. We employ the \cite{cox1979option} binomial tree method to value the option.
"Our function takes the following parameters as input: "
\begin{itemize}
    \item $S$: spot price of the underlying firm stock
    \item $K$: strike price of the option. In our case, we always set it equal to S (fn: some propose to set it slightly higher ...)
    \item $T$: time to maturity (in years)
    \item $v$: vesting period of the option (in years)
    \item $r$: risk-free interest rate
    \item $m$: exercise multiple
    \item $sigma$: volatility of the underlying firm stock
\end{itemize}
    
    It is practice to also specify the type of option (call or put) which we do not do since they are all call options and compute directly the corresponding payoff.



%Show table with simulations at different n
%Simulate also with trinomial model




\subsection{Executive value}
%Subjective value

%Build up on previous code
%%% Can re-use the stock price movements (both binomial and trinomial ways)
%%% Calculate at each node the utility from max{exercise, do NOT exercise}
%%%%%% -> find the utility value at initial node through recursion
%%% Using the inverse of the utility function, find the cash level that yields the same utility






\subsection{Comparative statics}
%Simulations to compare the two values




* effect of time vesting?? "time vesting has a relatively small impact on valuation but may dramatically affect the optimal exercise policy" \cite{dybvig2003employee}