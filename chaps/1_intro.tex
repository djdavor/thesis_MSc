\section{Introduction}
Employee Stock Options (ESOs) have been one of the most used forms of executive compensation since their introduction several decades ago. While this type of options initially appeared to be a straightforward mechanism for aligning employee and shareholder interests, their usage has been diverse, incorporating different characteristics and terms. This thesis focuses specifically on a subtype of ESOs known as reload options, with a particular emphasis on Dynamic Employee Stock Options. These options involve both an outright payment and the issuance of new reloaded options upon exercise. The objective is to investigate how these ESOs diverge from traditional ones and whether they can effectively accommodate executives with varying levels of risk aversion. To address this inquiry, we develop a simple model featuring a representative firm and heterogeneous executives, which we analyze from two perspectives: a more qualitative one, which builds on options' valuation, and a more quantitative one, by means of numerical simulations.


The thesis is organized as follows.
In Chapter 2, we provide an overview of the relevant literature streams, including ESOs and reload options, their valuation methodologies, risk aversion considerations, and principal-agent problems. We commence by discussing the historical development of ESOs and their role in executive compensation research. Notably, ESOs have increasingly exhibited heterogeneity in terms of their features and terms, leading to a wide array of studies. We subsequently shift our focus to reload options, which grant new options alongside the customary exercise compensation. Specifically, we explore the concept of Dynamic Employee Stock Options, a type of reload option proposed as a hybrid instrument that allows for the recapture of the time value associated with traditional ESOs, benefiting both executives and firms. Additionally, we delve into the valuation of traditional and reload ESOs, a critical aspect of the literature that sheds light on the incentives they provide to both firms and executives. Moreover, we delve into the theoretical literature on risk aversion and the principal-agent problem, particularly in dynamic settings, which forms the backdrop of our model.
Chapter 3 outlines the theoretical model employed throughout the thesis. We adopt a continuous-time principal-agent framework featuring adverse selection, represented by the executive's coefficient of risk aversion, and moral hazard, encompassing the executive's effort choice and volatility. In this model, a risk-neutral firm hires a risk-averse executive who can select between a traditional ESO, providing cash compensation upon exercise, and a distinct option that combines cash compensation with the issuance of new options upon exercise. The executive's risk aversion, effort level, and project volatility can vary, while the firm can determine the parameters $(\alpha, \gamma)$ that dictate the proportion of cash and options in the compensation package. The chapter concludes by highlighting how the problem evolves with the introduction of moral hazard and adverse selection, rendering it more realistic.
In Chapters 4 and 5, we explore the issue from two perspectives: valuation of the options in the former and equilibrium analysis through simple numerical simulations in the latter. In Chapter 4, we develop a novel valuation framework for the new option, accounting for both its recovered time value and additional compensation. An important observation pertains to the discrepancy in valuation between the firm and the executive due to the executive's risk aversion. While the traditional risk-neutral framework suffices for the firm, a utility-maximization approach reflecting the executive's risk aversion is necessary. We confirm the findings of previous research \citep{carpenter1998exercise}, suggesting that this discrepancy influences the valuations assigned by the firm and the executive. Furthermore, we delve deeper into the incentives provided by both options from both perspectives. Our findings indicate that the executive's incentive to exert effort is lower than anticipated by the firm for both options. This discrepancy is less pronounced for traditional ESOs, where the executive's incentive spikes when the option is just out-the-money or in-the-money, whereas the executive's incentive is consistently lower for the new option, irrespective of the level of risk aversion. With regard to volatility, while higher volatility increases the option's value from the firm's perspective, thereby increasing the incentive to opt for higher volatility, the executive has the incentive to decrease volatility, particularly so for the new option, which exposes her to higher fluctuations of the underlying stock.
In Chapter 5, we simulate the equilibrium of the model within a discrete-time framework. When the principal can observe both executive type and choices, all agents opt for the traditional option and exert high effort. However, in a more realistic scenario with unobserved types and actions, executives continue to select the traditional option but now with low effort and high volatility. These results contrast somewhat with the findings of the previous chapter but appear robust to parameter variations, except for the average number of years until exercise and the difference in effort values. The latter discrepancy arises due to limitations in our code's ability to handle very small results. Nonetheless, the disparity in the principal's utility between the two options remains modest, leaving room for future refinements of our model that may impact the outcomes.