ISSUES:
- Does the executive only affect through effort or also through volatility??






TO ADDRESS:
- Stock prices' movements only incorporate one DM (→ say that from one's perspective the other two parameters are exogenous - assuming NO correlation btw executives) 


EXTENSIONS:
- What happens if the agent is risk neutral too?
- What if it can affect the stock price only by exerting effort?
- Delayed effect of effort
- Observability and closeness 2nd and 1st best (?)
- Benchmark vs absolute (stock price) - exposing executives to benchmarks may reduce noise
- Modeling price as Levy process instead of geometric Brownian motion + allow for regime switches
- Stocks instead of call options → set S_0 = 0!
- The interpretation of this setting could possibly be extended also for example to the case of entrepreneur - funding agency (investors)
- Considering a replicating portfolio in terms of stocks with different exercise times?


COMMENTS:
- We wouldn't have early exercise if the agent was risk-neutral and there were NO dividends or voting rights in the stock -> would forgo time value of option; risk-averse instead does so to reduce correlation with firm's performance (diversification purposes) - he cannot hedge otherwise
- Whole compensation package is way larger usually: cash, options, restricted stock, performance bonuses
- Limitations: NO discounting, NO leaving for a different firm (may be represented by the reservation utility, even though it is not time-dependent → may be parabolic, due to increase in skills at beginning, then fading out bc of age)
- Strike price could be X_0, but can easily be extended to some other strike price (see paper ... arguing for higher strike price, so that option starts out-of-money)
- We want to understand the relationship btw ...
- LIMITATIONS: agent is tied indefinitely to principal (no quitting, firing dynamics); applies to public firms only, since the output process is the stock price; NO ESO repricing (which is sometimes done in periods of downturns, with options deep down out-the-money so to "revive" their incentive characteristic)
- Low vs high volatility in observability: in low case, sell firm to agent; in other cases, performance-incentives (eg, options) work really well
- It is easier to study the problem in continuous rather than discrete time (!)