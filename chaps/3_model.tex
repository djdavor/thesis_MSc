\section{Theoretical model}
Consider a continuous-time principal-agent model (with finite horizon) with both moral hazard and adverse selection.
We build up on the idea of the continuous-time principal-agent model pioneered by Sannikov (\cite{sannikov2008continuous}) and on the work on screening using ESOs by Cadenillas, Cvitanić and Zapatero (\cite{cadenillas2002executive}, \cite{cadenillas2005executive}). The difference from the latter is that we consider executives that are heterogeneous in their risk aversion.
%%Difference from the latters: from the first, ...; from the second, ...

We assume that markets are complete and frictionless, i.e., there are no taxes nor transaction costs - the former is a strong assumption but necessary to keep the problem simple.\footnote{Note that the agents' optimal policies may be otherwise influenced by the incidence of taxes on wealth. For example, an executive may anticipate or postpone the exercise of options due to changes in taxation which, despite being of interest, is out of our scope.}

Consider a risk-neutral firm and a risk-averse executive. Suppose the executive knows her own type $\rho$, while the firm only knows the distribution of types $\lambda$. We illustrate the main components of our model and then summarize the problem faced by the agents.


\subsection{Stock Price}
We consider a publicly listed firm. A standard geometric Brownian motion process $Z = \{ Z(t), \mathscr{F}_t \}_{t \ge 0}$ on a probability space $(\Omega, \mathscr{F}, P)$ drives the stock price. 
Therefore, the stock price $S(t)$ at time $t$ evolves according to 
$$ dS_t = \mu S_t dt + \sigma S_t dW_t $$
which can be re-written in the more familiar
$$ \frac{dS_t}{S_t} = \mu dt + \sigma dW_t $$
with starting value $S_0$. The process $W$ is a standard Brownian motion, $\mathscr{F}$ is the filtration generated by the Brownian motion process, $\mu$ is the exogenous drift, and $\sigma$ is the exogenous stock volatility. 

When the firm is managed by the executive, the dynamics of the stock price S are given by 
$$ dS_t = \mu S_t dt + a_t S_t dt + \delta \sigma_t S_t dt + \sigma S_t dW_t $$
which, similarly to before, and assuming $\mu = 0$, can be re-written as 
$$ \frac{dS_t}{S_t} = a_t dt + \delta \sigma_t dt + \sigma dW_t $$
where $a = \{a_t\}_{t \ge 0}$ and $\sigma = \{\sigma_t\}_{t \ge 0}$ are two adapted stochastic processes and $\delta \ge 0$ is a constant. The process $a$ represents the effort exerted by the executive and is such that $a(t) \ge 0 \forall t \ge 0$, while $\sigma$ is the choice of volatility level. While the interpretation of the former is straightforward, the latter is justified by the fact that we assume the executive could possibly face a menu of projects with different volatilities - not necessarily with different expected return - so that this choice is ultimately reflected also in the stock price. 
The interpretation behind $\delta$ is more subtle and taken from \cite{cadenillas2005executive}: it basically serves as a control for the impact of a project on the firm's overall stock volatility. For example, smaller firms may exhibit higher $\delta$ because they have less executives and hence each project matters, while for larger firms the impact of a single project on their stock price may be more limited. Or, we can interpret it as the \textit{relevance} of the executives, that is, when hiring C-level executives the $\delta$ would be higher - since they are expected to have a larger impact on the company trajectory - compared to when hiring middle managers, for example.

Few observations are in order. First, it is clear that the stock price is publicly observable by all parties, but its components are not. Therefore, the firm cannot perfectly observe the effort of the executive nor differentiate between the stock's intrinsic volatility and the part of volatility generated by the project's choice. Second, we do not distinguish between systematic and idiosyncratic volatility of the stock, which would be of interest for executive's hedging\footnote{Indeed, executives are often prohibited from trading the firm's stock on the market, prohibiting thus a hedge against their ESOs by shorting the stock. Therefore, the executive can only hedge against the idiosyncratic portion, which makes this problem even more interesting} but not relevant for our case. Finally, we are clearly in a partial equilibrium setting: if this was not the case, the price of the stock would have already incorporated all the possible information in the economy and there would be no room for influence from executives' decisions, making our analysis useless.


\subsection{Two ESOs}
Consider an American call option with maturity T (usually 10 years - see Marquardt (2002)), strike price $K$ 
For simplicity, we set $K=S_0$. The vesting period is denoted by $t_v$ (usually 2 to 4 years); in any case, $t_v < T$.
Vesting effects and what happens if employee leaves the firm ...

The vesting period is key - otherwise clearly both agents would ask for R one, since they could in theory exercise it immediately and no wealth-maximizing agent would choose the RN option.
We constrain the options to be exercised simultaneously, which simplifies our problem. 
We have already discussed that a traditional ESO contract prohibits the agent both from selling the option and hedging the risk
    %some extensions allow for selling it after some time though, so to be able to recover also the time value of the option
    it is also this feature that makes the analysis of executive's risk aversion relevant
A contract can be represented by (n, K), with n being the number of options granted at exercise price K. (case K = 0 corresponds to restricted stock)


\subsection{Executive}
(he, agent) risk-averse and effort-averse executive 
The executive can affect the price either by exerting effort (in the form of added drift component), which causes her disutility, or by choosing the volatility of the projects (in the form of added volatility component). Therefore, the maximi
Incentives: delta (/ vega in this case)
Reservation utility - same for both agents (simplifying assumption)
Disutility increasing and convex function; utility function follows usual concave representation of risk aversion
Assume the executive is tied forever to the principal
(Law of motion of continuation value (??))
(Agent's limited liability requires positive consumption)
CARA (Constant Absolute Risk-Aversion) preference: $u(c_t, a_t) = -\frac{1}{\rho} \exp^{-\rho c_t} - g(a_t)$, with $\rho>0$ being the absolute risk-aversion coefficient.
Optimal stopping time 
    show result from standard stopping theory
    NOT of interest here → we'll take it almost for granted, as if it was exogenous → would involve some complex modeling
    we can just assume that the more risk-averse agent will exercise earlier, ie, $t_v^{\rho_H} < t_v^{\rho_L}$
        %this can be probably proved formally by exploiting definition of stopping time + shape of utility function
Graph: Relationship strike price and utility of agent
Show with brackets how the pay/wealth of the agent evolves for different T's (before/after the exercises) \& when the options(s) are in/out the money


\subsection{Firm}
(she, principal)
The firm may offer two different compensation schemes: apart from cash, it offers a portion of compensation in the form of ESOs. We focus only on the latter portion. The firm grants ESOs as a performance incentive scheme
It is risk neutral hence maximizes the expected value




\subsection{Problem}
To summarize, we are analyzing the problem of a risk-neutral company that can offer two types of ESOs as a form of compensation to hire a risk-averse executive. The coefficient of risk aversion for a given executive is unknown but its distribution is common knowledge.
Recap of the two problems
-> formally, it is nothing more than a constrained maximization problem!
%Assumptions
Continuous incentive compatibility constraints
Participation constraints may NOT be binding - for effort reasons
%Explain the conditions...
The screening may happen bc of two reasons: (i) a more risk-averse agent could be made worse off by receiving more options (e.g., consider the case for $\gamma \rightarrow 0$ and the result from Stocks or Options? (..)) and (ii) risk from the underlying and the vesting period.

%Some comments are in order: ...
%We assume strong competitive forces so that neither party has relevant bargaining power -> NO need for modeling them 