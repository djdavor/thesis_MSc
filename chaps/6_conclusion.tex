\section{Conclusion}


\subsection*{Comments on the results}
In this chapter we have presented the 


The choice of including only the proportion of agents that choose the RN and the R option respectively in the principal's utility, is given by the fact that it allows us from to isolate from the impact of the number of executives. Indeed, the parameter $\beta$ and $\delta$ can already be used to control for the size of the firm, hence it would be repetitive. Therefore, the impact of the number of agents is limited to the stock price path, when some of them exert high effort.

$a_L> 0$ and $y_{R1} > y_{RN}$ have greatest impact. changing $a_H$ and $a_L$ to sth closer to each other would be interesting.
RIP competitive screening
    Screening happens with shutdown equilibrium, eg, 50/50 and $a_L, a_H = 0.5, 2$ but bc for low agent it is shutdown → but it is NOT incentive compatible!

    High risk-averse is better off in third best
%Consider wealthy CEO vs manager at some Internet-based firm (\cite{carpenter1998exercise}) - the former has lower portion of wealth invested in options hence has the "risk-adjusted" risk-aversion lower than the latter. Alternatively, differences in valuation are reflective of differences in wealth composition rather than risk aversion alone!! \cite{henderson}  

The qualitative discussion of (...) on DESOs remains valid, if we were to consider dynamics not fully encompassed by our model. Indeed, the only benefit of DESOs in our model is through an increase in the stock price thanks to a higher effort exerted by the effort, but in reality the gains may be more immaterial. As they discuss, they can include ...

We have assumed constant effort and constant volatility throghout the lifetime of the option; it is more realistic that the choice of volatility is not perfectly synchronized with the end of option, as well as effort cannot be taken for granted but rather should focus on instantaneous incentive to exert effort, which however would require a more complex recursive formulation and specification of continuation utility. Therefore, a fully fledged theoretical model here should be solved including features such as 

Subjective valuation discussion lead to: subjective incentive to exert effort is NOT always lower and it could be more profitable to grant options slightly ITM for (immediate) incentive purpose; BUT it is not lower only for RN option, whereas for the R option it is indeed lower.
+ contrasting results between two approaches: for RN option, in the valuation framework, incentive to increase volatility is negative; in the equilibrium analysis, incentive to increase volatility is positive. Unclear why...
    BUT R analysis limited due to computational constraints...

\subsection*{Concluding Remarks}
%short summary of chapters 
"We have a risk-averse executive that can choose the effort and volatility of the project, and a risk-neutral firm that can structure the R contract. The executive can be of low or high risk aversion, and her utility is given by the expected utility of wealth, discounted at the rate r > 0. The firm’s utility is given by the expected value of the terminal stock price minus the cost of compensating the agent with the two options. The firm offers two contracts to the agent, one including the RN option and the other the R option. The firm chooses how to structure the R option, that is, he chooses the values of $\alpha$ and $\gamma$. The firm’s problem is to maximize the expected profit, subject to the executive’s participation constraints."
Then, we first studied the cost of the two options and their sensitivities, both from the firm's and the executive's standpoints. This allowed us to ...
Finally, we run some numerical simulations, which do not ...; findings are robust to changes in specifications of the main parameters..; limitations of this methodology that do not allow to explore fully more RA agents

%%Does the utility-maximizing approach yield a stronger difference in valuation between the two options for the R option?


\textit{To be written.}

%IDEAS
    %Since the principal loses a lot of utility, we may think of some mechanism through which can induce agent to exert higher effort; but how to make it IC?

    %In finance, papers assume even higher levels of risk aversion, which we cannot use because of precision of our model.
    %BUT since increasing levels of risk aversion increase the pool of third best equilibria, maybe this gives hope if we also had more than 2 agents?

    %More realistic scenarios would allow for more than 2 types of agents, and more than 2 options. This would allow for more complex scenarios, and more realistic ones. But they would require a more complex machinery

    %We do NOT have expectations embedded in the model: the agent does not account for what the other agents will be doing

    %BUT the whole literature seems dependent on this heavy assumptions on ptf composition of agents and coefficients of risk aversion. This is a limitation of the literature, and of our model.

    



%WHY it matters (emphasize more in intro!)?
