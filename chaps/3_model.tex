\section{Theoretical model}
Consider a continuous-time principal-agent model (with finite horizon) with both moral hazard and adverse selection.
We build up on the idea of the continuous-time principal-agent model pioneered by Sannikov (\cite{sannikov2008continuous}) and on the work on screening using ESOs by Cadenillas, Cvitanić and Zapatero (\cite{cadenillas2002executive}, \cite{cadenillas2005executive}). The difference from the latter is that we consider executives that are heterogeneous in their risk aversion.
%%Difference from the latters: from the first, ...; from the second, ...

We assume that markets are complete and frictionless, i.e., there are no taxes nor transaction costs - the former is a strong assumption but necessary to keep the problem simple.\footnote{Note that the agents' optimal policies may be otherwise influenced by the incidence of taxes on wealth. For example, an executive may anticipate or postpone the exercise of options due to changes in taxation which, despite being of interest, is out of our scope.}

Consider a risk-neutral firm and a risk-averse executive. Suppose the executive knows her own type $\rho$, while the firm only knows the distribution of types $\lambda$. We illustrate the main components of our model and then summarize the problem faced by the agents.


\subsection{Stock Price}
We consider a publicly listed firm. A standard geometric Brownian motion process $Z = \{ Z(t), \mathscr{F}_t \}_{t \ge 0}$ on a probability space $(\Omega, \mathscr{F}, P)$ drives the stock price. 
Therefore, the stock price $S(t)$ at time $t$ evolves according to 
$$ dS_t = \mu S_t dt + \sigma S_t dW_t $$
which can be re-written in the more familiar
$$ \frac{dS_t}{S_t} = \mu dt + \sigma dW_t $$
with starting value $S_0$. The process $W$ is a standard Brownian motion, $\mathscr{F}$ is the filtration generated by the Brownian motion process, $\mu$ is the exogenous drift, and $\sigma$ is the exogenous stock volatility. 

When the firm is managed by the executive, the dynamics of the stock price S are given by 
$$ dS_t = \mu S_t dt + a_t S_t dt + \delta \sigma_t S_t dt + \sigma S_t dW_t $$
which, similarly to before, and assuming $\mu = 0$, can be re-written as 
$$ \frac{dS_t}{S_t} = a_t dt + \delta \sigma_t dt + \sigma dW_t $$
where $a = \{a_t\}_{t \ge 0}$ and $\sigma = \{\sigma_t\}_{t \ge 0}$ are two adapted stochastic processes and $\delta \ge 0$ is a constant. The process $a$ represents the effort exerted by the executive and is such that $a(t) \ge 0 \forall t \ge 0$, progressively measurable with respect to $\mathscr{F}$, while $\sigma$ is the choice of volatility level. While the interpretation of the former is straightforward, the latter is justified by the fact that we assume the executive could possibly face a menu of projects with different volatilities - not necessarily with different expected return - so that this choice is ultimately reflected also in the stock price. 
The interpretation behind $\delta$ is more subtle and taken from \cite{cadenillas2005executive}: it basically serves as a control for the impact of a project on the firm's overall stock volatility. For example, smaller firms may exhibit higher $\delta$ because they have less executives and hence each project matters, while for larger firms the impact of a single project on their stock price may be more limited. Or, we can interpret it as the \textit{relevance} of the executives, that is, when hiring C-level executives the $\delta$ would be higher - since they are expected to have a larger impact on the company trajectory - compared to when hiring middle managers, for example.

Few observations are in order. First, it is clear that the stock price is publicly observable by all parties, but its components are not. Therefore, the firm cannot perfectly observe the effort of the executive nor differentiate between the stock's intrinsic volatility and the part of volatility generated by the project's choice. Second, we do not distinguish between systematic and idiosyncratic volatility of the stock, which would be of interest for executive's hedging\footnote{Indeed, executives are often prohibited from trading the firm's stock on the market, prohibiting thus a hedge against their ESOs by shorting the stock. Therefore, the executive can only hedge against the idiosyncratic portion, which makes this problem even more interesting} but not relevant for our case. Finally, we are clearly in a partial equilibrium setting: if this was not the case, the price of the stock would have already incorporated all the possible information in the economy and there would be no room for influence from executives' decisions, making our analysis useless.


\subsection{Two ESOs}
We consider as ESO an American call option with maturity T - usually 10 years (\cite{marquardt2002cost}) - with strike price $K$. For simplicity, we set $K=S_0$. The vesting period is denoted by $t_v$ (usually 2 to 4 years); in any case, $t_v \le T$.\footnote{Note that when $t_v = T$ we have an European option.} Therefore, the executive cannot exercise the option when $t \in [0, t_v)$, but is free to do so when $t \in [t_v, T]$. The choice of the inclusion of $t_v$ is immaterial. We do not model explicitly the possibility of the executive leaving the firm; however, we assume that if this happens during the vesting period she receives zero payment, while if this happens after $t_v$ she will exercise the ESO only if it is in-the-money. %[fn] some model this

Now, we assume the firm can only offer two types of ESOs. We call the first one Risk-Neutral (RN) and the second one Risky (R):
\begin{enumerate}
    \item RN: this is the usual ESO. The option cannot be exercised before $t_v$, after which the executive can exercise one unit of RN for one unit of firm stock. The option expires at time T; at expiry, the employee will exercise it if in-the-money. The payoff at any $t \in [0, T]$ is thus $\max \{0, S_t - K \} = (S_t - K)^+$. We denote a generic risk-neutral option RN with $RN (T, K, t_v)$.
    \item R: this is a modified version of ESO, that takes from the literature on ESOs with reload option and DESOs (....).
    Similarly to before, the option cannot be exercised before $t_v$. However, when exercised, one unit of R is converted into $\alpha \in (0,1)$ units of stock and $1 - \alpha + \gamma$ units of RN, with $\gamma > 0$. The new RN option will have strike price $S_\tau$, where $\tau$ is the time of exercise. Keeping $T, K, t_v$ fixed, we can identify a specific R option with $(\alpha, \gamma)$, which we can denote as $R_{\alpha, \gamma} (T, K, t_v)$. Note that for $\gamma = 1$, we have a proper ESO with reload option. %literature on this

    %The payoff is 
\end{enumerate}
Let us illustrate R with an example. Consider a Risky ESO with maturity at $T=10$ years, strike price $K=S_0=\$5$, vesting period $t_v=3$ years, $\alpha = 0.7$, and $\gamma=0.1$. We denote it with $R_{0.7, 0.1} (10, 5, 3)$. This means that the executive cannot exercise the option in the first 3 years. After the 3rd year, he can choose to exercise it at any point within the 10th year. Suppose he decides the exercise the option 2 years after the vesting period ended, that is, 5 years from the date of initial issuance. Assume that the stock price when he decides to exercise it is $S_5 = \$10$. Therefore, the executive buys $0.7$ of a stock at $5$ dollars, netting thus a profit of $0.7*(\$10-\$5)=\$3.5$ on the trade. Moreover, she receives $0.3+0.1=0.4$ worth of new $RN(10, 5, 3)$, which will thus expire at $T=15$ and be exercisable since year $5+3=8$ at strike price $S_5=\$5$. 
It is clear from this example why the vesting period is key for both options to make sense. Suppose by contra that there was no vesting period in neither option. Then, since $\gamma > 0$ and the stock price cannot clearly be negative, no rational executive would choose the RN because in the worst scenario they can decide to exercise simultaneously the R and the new RN they would obtain, obtaining $\alpha + (1-\alpha+\gamma)=1+\gamma > 1$, where the right side is the number of stocks they would get from the RN option. It is exactly this additional vesting period that increases the risk for the executive, driven by the stochastic process of the underlying stock. Holding this security is risky because it could as well be that the stock price goes under $\$5$, at least after the 8th year - when the vesting period of the RN ends - so that the option goes out-of-money and the executive will never exercise it, netting thus only \$3.5. On the contrary, if the executive held a RN option and exercised it similarly at year 5, she would have obtained a payoff of \$5. This loss of $\$1.5$ is the risk associated with exercising the RN rather than the R option, and represents the \textit{maximal downside} the executive could expect in this case; clearly, the upside is, at least in theory, infinite. The intuition is that a low risk averse executive would then be willing to take this risk, while a risk averse agent would prefer the \$5 payoff of the RN option, foregoing thus the potential of a larger profit. Note however that the agents do not know a priori - when they will need to choose which option they prefer - the extent of the potential \textit{maximal downside} associated to choosing the R over the RN at any time $t$. This, and the fact that a traditional ESO contract prohibits the agent both from selling the option and hedging part the firm-specific risk, makes the analysis of executive's risk aversion relevant.

For simplicity we have focused on the case of one option, assuming also that shares can be fractioned, but in reality we work with larger numbers - this does not however compromise the final result. We will anyways constrain the options to be exercised simultaneously, which simplifies our problem (and is consistent with the empirical findings on block exercise of ESOs%lit
). 

%some extensions allow for selling it after some time though, so to be able to recover also the time value of the option
%A contract can be represented by (n, K), with n being the number of options granted at exercise price K. (case K = 0 corresponds to restricted stock)


\subsection{Executive}
Consider a risk- and effort-averse executive (she). The executive affects stock price dynamics by exerting costly effort - which adds as the drift component of the stock price - and by choosing the project with the desired volatility, which is not costly and affects part of the volatility component of the stock. Therefore, the maximization problem of the executive involves both the choice of effort and volatility. 

The executive signs a contract with the principal, whereby she can choose if she prefers to get $n$ units of RN or of R options. By simplicity, normalize $n$ to simply being equal to 1 unit. %is this OK?? 
Both are set at strike price $K = S_0$. If the executive decides to exercise at time $\tau \in [0, T]$, she nets a profit of $S_\tau - K$. At time $\tau = T$, she will exercise if and only if $S_\tau > K$ - we use the strict inequality because exercising has always some small cost involved, be it also immaterial in the form of time spent.
\\Note we can see this alternatively as a payment that the agent receives. %from whommmm??
Therefore, we have a stream of payments $c = \{c_t\}_{t \in [0, T]}$; in the case of the RN option, there will be at most one $\bar{t}$ such that $c(\bar{t}) > 0$, that we can then identify uniquely as $\tau$, while in the case of the R option we may have at most two $\bar{t}_1, \bar{t}_2$ such that $c(\bar{t}_1), c(\bar{t}_2) > 0$. We denote them by $\tau_1, \tau_2$.
This trick may not seem very intuitive at first, but is perfectly legitimate and will allow us to solve the problem using tools from \cite{sannikov2008continuous} and \cite{cvitanic2013dynamics}. Note that agent's limited liability requires positive consumption at any time $t$, so that $c(\cdot)$ is a nonnegative function.
%%This payment needs NOT to be coming from the principal! It would assume that the agent is selling its stock to the company itself, BUT in reality the profit is given by (i) selling the stock to somebody else (NOT necessarily the firm, if not in the special case of stock buyback) at a higher price than (ii) the strike price at which it buys it from the firm.

Therefore, the executive's utility of payment is denoted by $u(c(t), \rho)$, where $u(\cdot)$ is increasing and concave. Normalize $u(0) = 0$ and let $v(\cdot) = u^{-1}(\cdot)$ denote the inverse function. Suppose that $\rho$ indicates the risk aversion of the executive and takes only two values in $\Theta = \{\rho_L, \rho_H \}$ - we abuse terminology and call executive of type $\rho_L$ risk-lover (she is \textbf{L}ow risk averse) and the executive of type $\rho_H$ risk-averse (she is \textbf{H}igh risk averse). Clearly, denoting the $\rho_L$ type as risk-lover does not mean she always prefers to be exposed to more risk rather than less, but simply that she is less risk averse than the other type. The executive knows her own type, but the firm does not. However, the distribution of types in the population $\lambda = \Prob(\rho = \rho_L)$ is common knowledge. Clearly, $\Prob(\rho = \rho_H) = 1 - \lambda$. Recall that risk aversion is relevant here because the employee cannot sell the ESO nor perfectly hedge against it, hence it remains exposed to at least the portion of firm-specific risk.
On the other hand, the agent chooses effort $a_t \in [0, a_M]$ and effort is costly: we denote the cost of effort by $g(a_t)$ such that $g(\cdot)$ is strictly increasing and convex ($g'(\cdot) > 0$ and $g''(\cdot)<0$). We normalize it so that the expected output given effort $a_t$ is $a_t$, and $g(0) = 0$. Therefore, if effort is not compensated in some way, the executive would not have the incentive to exert effort. As we will see in the next chapter, the incentive for exerting effort has been measured by the delta of the ESO: %see lit
the intuition is that the delta measures the sensitivity of the option to the changes of the underlying stock, and since the agent affects the drift of the stock with her effort, she will exert it insofar this translates into a better value of her option. Others propose to also look at the option's vega, for similar reasons because the executive can also affect the volatility of the stock. For what concerns the choice of volatility, we could see the projects as comparable in risk, since higher risk yields (proportionally) higher expected return; note that choice of volatility has no cost of effort for the executive, but it affects the expected value of the compensation package. Indeed, volatility can increase the value of the stock and hence the compensation package, but at the cost of exposing the executive to higher risk - both of these are further magnified by the intrinsic leverage effect of options \cite{cadenillas2005executive}.


Now, assuming that the agent discounts the utility at some rate $r>0$, the expected utility $U(\cdot)$ is given by
$$U(\rho_i, c_i, a_i) = r \EX \Bigg[ \int_{0}^{T} e^{-rs} \Big[u(c_i(s), \rho_i) - g(a_i(s)) \Big] ds \Bigg] $$
The executive has a reservation utility $R \ge 0$, which will play a role when solving the problem. We assume both types of agents have the same reservation utility. %DO WE?


%CARA (Constant Absolute Risk-Aversion) preference: $u(c_t, a_t) = -\frac{1}{\rho} \exp^{-\rho c_t} - g(a_t)$, with $\rho>0$ being the absolute risk-aversion coefficient.
Note that in our basic formulation we assume the executive is tied forever to the principal once he accepts the contract, or at least until time T. Moreover, we are considering an extremely simplified model: in practice, the executive usually holds ESOs with different strike prices and vesting periods as they are granted over time as the job contract goes on. In our setting, we are only considering the effect of the grant of one ESO, which is the only one the agent receives up to time T. 



%Optimal stopping time 
    %show result from standard stopping theory
    %NOT of interest here → we'll take it almost for granted, as if it was exogenous → would involve some complex modeling
    %we can just assume that the more risk-averse agent will exercise earlier, ie, $t_v^{\rho_H} < t_v^{\rho_L}$ -- CAN WE??
        %this can be probably proved formally by exploiting definition of stopping time + shape of utility function
%Graph: Relationship strike price and utility of agent



\subsection{Firm}
(he, principal)
The firm may offer two different compensation schemes: apart from cash, it offers a portion of compensation in the form of ESOs. We focus only on the latter portion. The firm grants ESOs as a performance incentive scheme
The company compensates the agent with ESOs
It is risk neutral hence maximizes the expected value


Time horizon of the firm matches the above described one


\subsection{Problem}
To summarize, we propose a principal–agent problem of moral hazard and adverse selection in continuous time whereby ... we are analyzing the problem of a risk-neutral company that can offer two types of ESOs as a form of compensation to hire a risk-averse executive. The coefficient of risk aversion for a given executive is unknown but its distribution is common knowledge.
Recap of the two problems
-> formally, it is nothing more than a constrained maximization problem!
%Assumptions
Continuous incentive compatibility constraints
Participation constraints may NOT be binding - for effort reasons
%Explain the conditions...
The screening may happen bc of two reasons: (i) a more risk-averse agent could be made worse off by receiving more options (e.g., consider the case for $\gamma \rightarrow 0$ and the result from Stocks or Options? (..)) and (ii) risk from the underlying and the vesting period.
IC: the agent may misreport -> temptation value
shutdown contract (?)
The problem is difficult and interesting because (i) it needs to give instantaneous incentives for the agent to work + (ii) the aggregate incentives for the agent to "reveal" its true type at time 0 need to be sufficiently good
Two state variables: continuation value + temptation-value process
→ optimal control, optimal stopping → derive then equilibrium utilities

--> screening allows to offer a contract to both agents and make the risk-averseless agent exert higher effort (see next chapter)

Show with brackets how the pay/wealth of the agent evolves for different T's (before/after the exercises) \& when the options(s) are in/out the money



%Some comments are in order: ...
%We assume strong competitive forces so that neither party has relevant bargaining power -> NO need for modeling them 