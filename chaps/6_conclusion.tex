\section{Conclusion}

\subsection{Discussion of Results}
In the previous chapter we have presented the main results arising from numerical simulations on the theoretical model built in Chapter 3. We now discuss the main findings and limitations of this approach. 
We find that the R option does not seem to play a role in any equilibrium setting, since even when it is not the trivial $R_{1,0}$ the agents would rather choose the RN option. Moreover, it does not help the firm in inducing any agent to exert high effort and in no scenario we obtain a true competitive screening, whereby the high risk averse agent chooses the RN option and the low risk averse agent the R option, and both exert high effort. The only scenario where this happens is in the case of a shutdown equilibrium, where one agent chooses the reservation utility and is not employed\footnote{This happens for example in the case of 50/50 executives, when setting $a_L, a_H = 0.5,2$ and holding the other parameters as usual.}, which is not however the competitive screening we had in mind.
The introduction of moral hazard first, and adverse selection then, induces the agents to choose low effort and high volatility and makes the high risk-averse comparatively better off than the low risk-averse, when compared to the first best outcomes. As for the robustness, either setting $a_L > 0$ or $y_R1$ higher than $y_{RN}$ have the greatest impact on the results: while for the former it seems sensible to think that low effort does not mean zero effort, the latter is more puzzling and not backed up by strong empirical evidence. For the latter, a different approach would be to model optimal exercise via optimal stopping theory and solve the whole model algebraically, but we are not sure if a closed-form solution exists. It would also be important to study if these results change when making $a_H$ and $a_L$ arbitrarily close, which could be a possible solution to this puzzle, but it is not allowed by the computational constraints of the model. This relates also to the fact that we cannot test the model numerically on risk aversion coefficients higher than $3$ and that in the valuation chapter we need to limit our analysis on the R option, due to these computational constraints.
The main limitation of the model is, as mentioned before, that effort and volatility are assumed to be constant during the lifetime of the option: clearly, it would be more realistic if the choice of volatility was not perfectly synchronized with the end of option, as well as if the effort was not taken for fixed, but rather studied in terms of maintaining instantaneous incentive during the entire lifetime of an option. This would however require an even more complex and recursive formulation of the problem, along with the introduction of continuation utility and stochastic machinery. Moreover, we do not account for the fact that the principal may have a preference over projects and that the agent may have expectations about the future, both in terms of stock price and of the controls the other executives choose. These limitations are not however present in the valuation framework, which could thus seem a more sensible method, also in light of the divergences between the two. As discussed in the fourth chapter, the valuation methodology predicts that the subjective incentive to exert effort is not always lower than the objective one, at least for the RN option, and that it could thus be more profitable to grant options slightly OTM for incentive purposes. But, it also predicts results that are in contrast with the numerical simulations, such as the positive incentive to increase effort and negative for volatility. However, this could be due to the fact that the valuation approach does not account for the costs of effort, which are instead accounted for in the numerical model. But this explanation does not hold true for the volatility, as the only cost is the higher exposure to fluctuations, which is accounted for also in the valuation approach and predicted to be low. Therefore, we do not always obtain convergence between the two approaches.

\subsection{Concluding Remarks and Future Research}
%Short summary of chapters 
In this thesis we have studied the problem of a firm that pays his risk-averse executives using employee stock options with an enhanced reload feature. We have built a theoretical framework to study the problem, that we have then analyzed using two approaches: (i) a valuation framework and (ii) numerical simulations. 
For the former, we first developed a novel valuation framework for the R option, which is able to account for the recovered time value at exercise. Then, we have analyzed the difference in firm cost and executive's subjective valuation: we find that the risk-averse executive values the option lower than the firm, the incentive to exert effort is lower than what the firm would predict, and that she would choose the project with lower volatility, even if this would seem to decrease the value of the option.
%The firm is better off since this decreases the cost of the option!!
These results are somewhat in contrast with the numerical simulations, where, instead, we find that executives always choose the RN option and high volatility, and low effort when we introduce moral hazard. These results rely on assumptions on the values of the two effort levels and average years before exercise, which impact the results but do not overturn them. However, the difference in principal's utility between the two options is never very high, which leaves open the possibilities for future refinements of our model, along with extensions to more than two types and two options. 
This, along with the relevance of the usual assumptions in the literature on risk aversion coefficients and portfolio composition, which are usually taken as given with limited empirical support but influence significantly the results, call for higher empirical research on these issues.