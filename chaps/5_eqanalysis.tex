\section{Equilibrium analysis}

*Contracts are history dependent, BUT we write them recursively with few \#variables.
*FOA: in static setting this allows to go past the IC constraints and solve the Lagrangian. Sufficient conditions in static setting are MLRP (monotone likelihood ratio property) and CDFC (convexity of distribution function condition). In dynamic the necessary conditions may be sufficient when the Hamiltonian is concave.
    BUT not always valid; if not, check validity ex-post; conditions are necessary only to know in advance if it is gonna be valid




ONLY moral hazard:
- recursive representation conditions only on "promised"/continuation utility; when introducing also moral hazard, we need to condition on sth more






Goal: show how allocations change within the three different equilibria: full info, moral hazard, moral hazard + adv. selection 



%Optimal exercise: in the form of exercise policy, which is done by studying the so-called continuation region of the employee (ie, where the employee would continue to hold the option)


\subsection*{Choice of parameters}
\cite{dai2005valuing} propose using N=10,000 as a good asymptotic approximation.




\subsection{Different executives}
Consider wealthy CEO vs manager at some Internet-based firm (\cite{carpenter1998exercise}) - the former has lower portion of wealth invested in options hence has the "risk-adjusted" risk-aversion lower than the latter. Alternatively, differences in valuation are reflective of differences in wealth composition rather than risk aversion alone!! \cite{henderson}  